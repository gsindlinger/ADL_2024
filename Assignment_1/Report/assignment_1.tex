\documentclass[12pt,a4paper]{article}
\usepackage{microtype}
\usepackage{lmodern}
\usepackage{mathpazo}
\usepackage{latexsym}
\usepackage{amsmath,amssymb,amsthm}
\usepackage[english]{babel}
\usepackage{float}
\usepackage{graphicx}
\usepackage{hyperref}
\usepackage[utf8]{inputenc}
\usepackage{listings}
\usepackage{xcolor}
% use more of the page
\usepackage[scale=0.8]{geometry}
% do not move figures across sections
\usepackage[section]{placeins}

% Sensible defaults for lstlistings
\lstset{
  basicstyle=\footnotesize\ttfamily,
  belowcaptionskip=1\baselineskip,
  breaklines=true,
  commentstyle=\bfseries\color{purple!40!black}
  frame=L,
  identifierstyle=\color{blue},
  keywordstyle=\bfseries\color{green!40!black},
  language=python,
  showstringspaces=false,
  stringstyle=\color{orange},
  xleftmargin=\parindent,
}
% prettier links
\hypersetup{
colorlinks,
linkcolor={red!50!black},
citecolor={blue!50!black},
urlcolor={blue!80!black}
}
\urlstyle{rm}

\begin{document}
\title{Advanced Deep Learning 2024\\Assignment 1}
\author{\color{red}Your name}
\maketitle

\section{Sobel filter}
\begin{figure}
  \begin{center}
%    \includegraphics[width=.75\textwidth]{Assignment1_Question6_Plot1}
  \end{center}
  \caption{Every figure and table has a caption and is referred to 
    in the text body.\label{fig:q6p1}}
\end{figure}



You can add code like this
\begin{lstlisting}
c = conv(x)
c = torch.square(c)
\end{lstlisting}

For combining the filter outputs, you can  use 
\lstinline{torch.sum}, but you have to set the \lstinline{axis}
argument properly.

\section{Convolutional neural networks}

no problem if you do or do not see the activation function when you output of the architecture 


\subsection{Basic CNN definition}
\subsection{Activation function before or after pooling?}
\subsection{Data augmentation}
\subsection{Batch normalization}
\subsection{Experimental architecture comparison}
\section{Fully convolutional neural networks}
\subsection{Receptive field and double convolution}


\begin{center}
\begin{tabular}{lcccc}
     &  $d=1$ &  $d=2$ &  $d=3$ & general $d$\\
 number of trainable parameters   &$n$& $n^2$& $n^3$ & $n^d$\\
 receptive field size  &$n$& $n^2$& $n^3$ & $n^d$
\end{tabular}
    \end{center}


\end{document}